\documentclass[11pt]{article}
\usepackage{amsmath}
\usepackage{xeCJK}
\usepackage{amsfonts}
\usepackage{color}
\usepackage{hyperref}
\usepackage{listings}

\lstset{frame=tb,
	language=Python,
	aboveskip=3mm,
	belowskip=3mm,
	showstringspaces=false,
	columns=flexible,
	basicstyle={\small\ttfamily},
	numbers=none,
	numberstyle=\tiny\color{gray},
	keywordstyle=\color{blue},
	stringstyle=\color{mauve},
	breaklines=true,
	breakatwhitespace=true,
	tabsize=4
}

\begin{document}
\section{DFT}
DFT其实就是使用有限个数的基$\{1, e^{ix}, ..., e^{i(N-1)x}\}$以及有限个数的插值节点$\{x_0, x_1, x_{N-1}\}, x_j=\frac{2\pi}{N}j$去逼近函数$f(x)$:
\begin{equation}\label{eq:1}
f(x) = \sum_{j=0}^{N-1}\hat{f}_je^{-ijx}
\end{equation}
很容易证明,这些基底在离散情形下也是正交的:
\begin{equation}
\langle e^{ikx}, e^{ilx}\rangle = \sum_{j=0}^{N-1}e^{i(k-l)\frac{2\pi}{N}j}=\begin{cases}
0 & k = l\\
N & k\neq l
\end{cases}
\end{equation}
那么根据正交性,结合公式(\ref{eq:1}),就能得到
\begin{equation}
\hat{f}_k = \frac{\langle f(x), e^{ikx}\rangle}{\langle e^{ikx}, e^{ikx} \rangle} = \frac{1}{N}\sum_{j=0}^{N-1}f_je^{-i\frac{2\pi}{N}j}
\end{equation}
其实前面的系数$\frac{1}{N}$都是无关紧要的,所以为了下面讨论的方便,我们直接写为
\begin{equation}
\hat{f}_k = \sum_{j=0}^{N-1}f_je^{-i\frac{2\pi}{N}j}
\end{equation}
\section{FFT}
从上一节可以知道,DFT其实就是矩阵与向量的乘积运算,算法复杂度为$O(n^2)$。能不能做到更好呢?事实上是可以的,启发就来源于乘法的结合律$ab + bc = a(b + c)$,等式左边进行了2次乘法操作,而等式右边其实只进行了1次乘法操作,于是计算的复杂度得以降低。所以,FFT就出现了,可以从后面的讲解中看到,其实FFT就是避免了大量的重复计算,然后把算法的复杂度降低到了$O(n\log n)$。

首先我们考虑要对一个长度为$2N$的序列做DFT,那么有:
\begin{equation}
\hat{f}_k = \sum_{j=0}^{2N-1}f_je^{-i\frac{2\pi}{2N}j}
\end{equation}
记$\bar{w} = e^{-i\frac{2\pi}{2N}}$,并且把下标为偶数和奇数的部分分割开来,那么有:
\begin{equation}
\hat{f}_k = \sum_{j=0}^{N-1}f_{2j}\bar{w}^{k(2j)} + \sum_{j=0}^{N-1}f_{2j+1}\bar{w}^{k(2j+1)}
\end{equation}
我们已经把奇数和偶数部分拆分开了,现在希望这两部分分别做一个长度为$N$的DFT。怎么做呢?我们记$\bar{W}=e^{-\frac{2\pi}{N}}$,那么容易推导出$\bar{W} = \bar{w}^2$,于是上面的式子可以进一步写为
\begin{equation}
\hat{f}_k = \sum_{j=0}^{N-1}f_{2j}\bar{W}^{kj} + \bar{w}^k\sum_{j=0}^{N-1}f_{2j+1}\bar{W}^{kj}
\end{equation}
容易观察出$\sum_{j=0}^{N-1}f_{2j}\bar{W}^{kj}$和$\sum_{j=0}^{N-1}f_{2j+1}\bar{W}^{kj}$就是分别对序列$\{f_0, f_2, ..., f_{2N-2}\}$和$\{f_1, f_3, ..., f_{2N-1}\}$做DFT,于是可以记为:
\begin{equation}
\hat{f}_k = F(f_0, f_2, ..., f_{2N-2})_k + \bar{w}^kF(f_1, f_3, ..., f_{2N-1})_k
\end{equation}
也就是说,我们可以从两个长度为$N$的子序列的DFT结果求得长度为$2N$的序列的DFT结果。有人会问了,其实这样做了以后复杂度还是没有降低啊?的确,我们还要推导一下$\hat{f}_{k+N}$与这两个子序列的关系,下面给出推导:
\begin{align}
\begin{split}
\hat{f}_{k+N} &= \sum_{j=0}^{N-1}f_{2j}\bar{W}^{(k+N)j} + \bar{w}^{(k+N)}\sum_{j=0}^{N-1}f_{2j+1}\bar{W}^{(k+N)j} \\
 &=\sum_{j=0}^{N-1}f_{2j}\bar{W}^{kj} + \bar{w}^{(k+N)}\sum_{j=0}^{N-1}f_{2j+1}\bar{W}^{kj}\quad(\bar{W}^{Nj}=1) \\
 &=\sum_{j=0}^{N-1}f_{2j}\bar{W}^{kj} - \bar{w}^{k}\sum_{j=0}^{N-1}f_{2j+1}\bar{W}^{kj}\quad(\bar{w}^N=-1)
\end{split}
\end{align}
现在有点意思了,我们发现其实$f_k$和$f_{k+N}$就只差了一个加减号而已。这样的话,我们做两个序列长度为$N$的DFT就有意义了:原来求{\color{red}一个}$\hat{f}_{k}$的时候大概要消耗$2N$的计算量,但是经过公式的变换以后求{\color{red}两个}$\hat{f}_k$才消耗了$2N$的计算量(每个序列长度为N的计算消耗$N$)。哈哈,是不是很神奇!

当然,上面的分析只是把$2N$的序列分成两个$N$的序列,我们可以接着这样做下去,$N\to\frac{N}{2}, ..., 4\to2$。于是就有了著名的快速傅里叶变换FFT。现在我们来分析一下FFT的时间复杂度:
\begin{equation}
T(N) = 2T(N/2) + O(N)
\end{equation}
$2T(N/2)$为计算两个序列长度为$N/2$的FFT时间,得到了这两个序列的FFT以后,我们可以重构序列长度为$N$的FFT,需花费$O(N)$时间。稍微学习过算法的同学都会知道,上面的递推关系给出的时间复杂度为
\begin{equation}
T(N) = O(N\log N)
\end{equation}
\section{Implementation of FFT}
其实如果序列长度是$2$次幂的话,每次分割都会很顺利,这时的FFT时间效率是最高的。自己实现了这种情形的FFT
\begin{lstlisting}
import numpy as np


def my_fft(y):
	"""
	Recursive implementation of FFT.
	
	:param y: np.array, with length equal to `2^L`.
	
	:return
	yh: np.array, same length as `y`.
	"""
	n = len(y)
	N = n // 2
	if n == 2:
		y0, y1 = y
		yh0 = y0 + y1
		yh1 = y0 - y1
		return np.array([yh0, yh1])
	yeven = my_fft(y[::2])
	yodd = my_fft(y[1::2])
	wk = np.exp(-np.arange(N)*np.pi/N * 1j)
	yh = np.zeros(n, dtype=np.complex128)
	yh[:n//2] = yeven + wk * yodd
	yh[n//2:] = yeven - wk * yodd
	return yh
\end{lstlisting}
\end{document}